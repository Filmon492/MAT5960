\section{Preliminaries}
\subsection{Scalar local and nonlocal conservation laws}

Scalar conservation laws are partial differential equations that can be written in the form
\begin{equation}\label{eq:scalar_conservation}
    \begin{split}
        \partial u + \nabla \cdot f(u) & = 0        \\
        u(x,0)                         & = u_{0}(x)
    \end{split}
\end{equation}
where $\partial_{t} = \frac{\partial}{\partial_{t}}$ is time differentiation, $\nabla \cdot$
is divergence operator, $ u = u(x,t)$ is the unknown and $f$ is a given flux function.

If we now let $f(u) = u(1-u)$, then \cref{eq:scalar_conservation} can be used for modelling traffic. For detailed setup of this model, we refer to
\cite{Ulrik2015}.
However, this model is based only on local information. Usually, when a car driver decides his speed, it depends on the traffic information within a road segment
of length $\epsilon > 0$ ahead of the cars's current location. For this to take into account we look for a better way to model
the velocity, hence we need a nonlocal model which is on the following form
\begin{equation}\label{eq:nonlocal_model}
    \begin{split}
        \partial u + \nabla \cdot (uV(u)) & = 0        \\
        u(x,0)                            & = u_{0}(x)
    \end{split}
\end{equation}
where now $V$ is a nonlocal operator, say,
\begin{equation}
    V(u)(x,t) = \int_{0}^{\epsilon} \omega_{\epsilon}(y) v(u(x+y,t)) \, dy
\end{equation}
for some
nonlocal kernel $\omega_{\epsilon}: [0,\epsilon] \rightarrow \mathbb{R}$ satisfying $\omega_{\epsilon} \geq 0$ and $\int_{0}^{\epsilon} \omega_{\epsilon}(y)  \, dy = 1$.
formally speaking, $V(u) \rightarrow v(u)$ as $\epsilon \rightarrow 0$, but whether the corresponding solutions
$u_{\epsilon}$ of \eqref{eq:nonlocal_model} converge to a solution of \eqref{eq:scalar_conservation}, is the big question that has been studied.

To sum this up, let us denote \eqref{eq:scalar_conservation} and \eqref{eq:nonlocal_model} as local and nonlocal model, respectively. Then we are aiming
to approximate solution of the local model by solutions of the nonlocal model, where the solution $u(x,t)$ denote the density of cars which is
number of cars per square meter.

\subsection{Solutions to the Riemann problem}
For numerical experiments we need entropy solutions for different initial data. Hence, we briefly mention the notion of weak solution
and entropy solution.

\begin{definition}[Weak solutions]
    A function $u \in L^{\infty}(\mathbb{R} \times \mathbb{R_{+}})$ is a weak solution of \eqref{eq:scalar_conservation} with initial
    data $u_{0} \in L^{\infty}(\mathbb{R})$ if the following identity holds for all test functions $\phi \in C^{1}(\mathbb{R} \times \mathbb{R_{+}})$
    \begin{equation}
        \int_{\mathbb{R_{+}}}^{} \int_{\mathbb{R}}^{} {
            u \phi_{t} + f(u) \phi_{x} \,dx \, dt
        }
        + \int_{\mathbb{R}}^{} u_{0} \phi(x,0) \,dx = 0.
    \end{equation}
\end{definition}
Since weak solutions are not unique, therefore some extra conditions have to be imposed in order to pick a weak solution
that describes the physical flow correctly. One such condition is entropy condition.
Detailed characteristics of entropy solution can be found on \cite{Ulrik2015_entropy}.

Now we look at the explicit solutions for the Riemann problem.
The Riemann problem is the initial value problem
\begin{equation}
    \begin{split}
        \partial_{t} u + \nabla \cdot f(u) & = 0 \\
        u(x,0)                             & =
        \begin{cases}
            u_{l} & \text{if } x < 0 \\
            u_{r} & \text{if } x > 0
        \end{cases}
    \end{split}
\end{equation}

Solution of this equation is constructed as
\begin{equation}
    u(x,t)=
    \begin{cases}
        u_{l}                  & \text{for } x < h'(u_l)t            \\
        (h')^{-1}(\frac{x}{t}) & \text{for } h'(u_l)t < x < h'(u_r)t \\
        u_{r}                  & \text{for } x > h'(u_r)t
    \end{cases}
\end{equation}
where $h(u)$ is defined by
\begin{equation}
    h(u) =
    \begin{cases}
        \sup\{ g(u) \mid g \leq f \text{ and } g \text{ is convex on } [u_l, u_r] \}  & \text{if } u_l < u_r, \\
        \inf\{ g(u) \mid g \geq f \text{ and } g \text{ is concave on } [u_r, u_l] \} & \text{if } u_r < u_l.
    \end{cases}
\end{equation}
